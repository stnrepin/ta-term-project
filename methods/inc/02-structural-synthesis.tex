% vim: set textwidth=120:

\section{Структурный синтез}

Проведем структурный синтез получившегося автомата Мили в базисе
$\left\{ \land, \lor, \lnot \right\}$ с использованием D-триггеров.

Закодируем входы и выходы двоичными числами:

\begin{table}[H]
\newcolumntype{M}{>{\centering\arraybackslash} m{3cm} }
\centering
\begin{tabular}{|M|M|}
    \hline
    Вход & $x_2x_1x_0$ \\
    \hline
    0  & 000 \\
    \hline
    1  & 001 \\
    \hline
    2  & 010 \\
    \hline
    3  & 011 \\
    \hline
    \$ & 100 \\
    \hline
\end{tabular}
\hfill
\begin{tabular}{|M|M|}
    \hline
    Выход & $x_2x_1x_0$ \\
    \hline
    0 & 000 \\
    \hline
    1 & 001 \\
    \hline
    2 & 010 \\
    \hline
    3 & 011 \\
    \hline
    i & 100 \\
    \hline
\end{tabular}
\end{table}

Закодируем состояния автомата метод минимизирующий число переключений элементов
памяти. Для этого сперва найдем вершину, имеющую наибольшую полустепень
захода, а также построим таблицу, в ячейках которой запишем число ребер между
$q_i$ и $q_j$:
\begin{table}[H]
\newcolumntype{M}{>{\centering\arraybackslash} m{1.5cm} }
\centering
\begin{tabular}{|M|M|}
    \hline
    Выход & $x_2x_1x_0$ \\
    \hline
    $q_0$ & 5 \\
    \hline
    $q_1$ & 3 \\
    \hline
    $q_2$ & 5 \\
    \hline
    $q_3$ & 7 \\
    \hline
\end{tabular}
\hfill
\begin{tabular}{|M|M|M|M|M|}
    \hline
          & $q_0$ & $q_1$ & $q_2$ & $q_3$ \\
    \hline
    $q_0$ &   2   &   2   &   2   &   2   \\
    \hline
    $q_1$ &   2   &   1   &   1   &   1   \\
    \hline
    $q_2$ &   2   &   2   &   3   &   1   \\
    \hline
    $q_3$ &   2   &   1   &   1   &   4   \\
    \hline
\end{tabular}
\end{table}

Из таблица видно, что вершина, имеющая наибольшую полустепень --- $q_3$:
\[
    q_3 \to 00
\]
Состояние наиболее связанное с $q_3$ --- $q_0$:
\[
    q_0 \to 01
\]
Состояние наиболее связанное с $\left\{ q_3, q_0\right\}$ --- $q_1$:

\vspace{-1cm}%
\begin{minipage}[t]{0.48\textwidth}
    \begin{table}[H]
        \centering
        \begin{tabularx}{\textwidth}{|C|C|}
            \hline
            Выход & $2d_{10} + 1\cdot d_{31}$ \\
            \hline
            10 & 5 \\
            \hline
            11 & 4 \\
            \hline
        \end{tabularx}
    \end{table}
\end{minipage}
\begin{minipage}[t]{0.48\textwidth}
    \vspace{1cm}
    \[
        q_1 \to 11
    \]
\end{minipage}

Оставшееся состояние --- $q_2$:
\[
    q_2 \to 10
\]

Построим таблицу входов и выходов для автомата Мили:
\begin{table}[H]
    \centering
    \begin{tabularx}{\textwidth}{|C|C|C|C|C|} \hline
            & 01 &  11 &  10 &  00 \\
        \hline
        000 & \slashbox{01}{100} & \slashbox{11}{100} & \slashbox{10}{100} & \slashbox{00}{100} \\
        \hline
        001 & \slashbox{11}{100} & \slashbox{11}{100} & \slashbox{10}{100} & \slashbox{00}{100} \\
        \hline
        010 & \slashbox{10}{100} & \slashbox{10}{100} & \slashbox{10}{100} & \slashbox{00}{100} \\
        \hline
        011 & \slashbox{00}{100} & \slashbox{00}{100} & \slashbox{00}{100} & \slashbox{00}{100} \\
        \hline
        100 & \slashbox{01}{000} & \slashbox{01}{001} & \slashbox{01}{010} & \slashbox{01}{011} \\
        \hline
    \end{tabularx}
\end{table}

Теперь постоим таблицу переходов и выходов:
\begin{table}[H]
\centering
\begin{tabularx}{\textwidth}{|C|CCC|C|CC|C|CCC|C|CC|CC|}
    \hline
    \small{Вх} & $x_2$ & $x_1$ & $x_0$ & $q_i$ & $Q_1$ & $Q_0$ & \small{Вых} & $y_2$ & $y_1$ & $y_0$ & $q_i'$ & $Q'_1$ & $Q'_0$
                    & $D_1$ & $D_0$ \\
    \hline
    0  & 0 & 0 & 0   &   $q_3$ & 0 & 0   &   i & 1 & 0 & 0   &  $q_0$ & 0 & 0   &   0 & 0   \\
    0  & 0 & 0 & 0   &   $q_0$ & 0 & 1   &   i & 1 & 0 & 0   &  $q_1$ & 0 & 1   &   0 & 1   \\
    0  & 0 & 0 & 0   &   $q_2$ & 1 & 0   &   i & 1 & 0 & 0   &  $q_2$ & 1 & 0   &   1 & 0   \\
    0  & 0 & 0 & 0   &   $q_1$ & 1 & 1   &   i & 1 & 0 & 0   &  $q_3$ & 1 & 1   &   1 & 1   \\
    \hline
    1  & 0 & 0 & 1   &   $q_3$ & 0 & 0   &   i & 1 & 0 & 0   &  $q_0$ & 0 & 0   &   0 & 0   \\
    1  & 0 & 0 & 1   &   $q_0$ & 0 & 1   &   i & 1 & 0 & 0   &  $q_1$ & 1 & 1   &   1 & 1   \\
    1  & 0 & 0 & 1   &   $q_2$ & 1 & 0   &   i & 1 & 0 & 0   &  $q_2$ & 1 & 0   &   1 & 0   \\
    1  & 0 & 0 & 1   &   $q_1$ & 1 & 1   &   i & 1 & 0 & 0   &  $q_3$ & 1 & 1   &   1 & 1   \\
    \hline
    2  & 0 & 1 & 0   &   $q_3$ & 0 & 0   &   i & 1 & 0 & 0   &  $q_0$ & 0 & 0   &   0 & 0   \\
    2  & 0 & 1 & 0   &   $q_0$ & 0 & 1   &   i & 1 & 0 & 0   &  $q_1$ & 1 & 0   &   1 & 0   \\
    2  & 0 & 1 & 0   &   $q_2$ & 1 & 0   &   i & 1 & 0 & 0   &  $q_2$ & 1 & 0   &   1 & 0   \\
    2  & 0 & 1 & 0   &   $q_1$ & 1 & 1   &   i & 1 & 0 & 0   &  $q_3$ & 1 & 0   &   1 & 0   \\
    \hline
    3  & 0 & 1 & 1   &   $q_3$ & 0 & 0   &   i & 1 & 0 & 0   &  $q_0$ & 0 & 0   &   0 & 0   \\
    3  & 0 & 1 & 1   &   $q_0$ & 0 & 1   &   i & 1 & 0 & 0   &  $q_1$ & 0 & 0   &   0 & 0   \\
    3  & 0 & 1 & 1   &   $q_2$ & 1 & 0   &   i & 1 & 0 & 0   &  $q_2$ & 0 & 0   &   0 & 0   \\
    3  & 0 & 1 & 1   &   $q_1$ & 1 & 1   &   i & 1 & 0 & 0   &  $q_3$ & 0 & 0   &   0 & 0   \\
    \hline
    \$ & 1 & 0 & 0   &   $q_3$ & 0 & 0   &   3 & 0 & 1 & 1   &  $q_0$ & 0 & 1   &   0 & 1   \\
    \$ & 1 & 0 & 0   &   $q_0$ & 0 & 1   &   0 & 0 & 0 & 0   &  $q_0$ & 0 & 1   &   0 & 1   \\
    \$ & 1 & 0 & 0   &   $q_2$ & 1 & 0   &   2 & 0 & 1 & 0   &  $q_0$ & 0 & 1   &   0 & 1   \\
    \$ & 1 & 0 & 0   &   $q_1$ & 1 & 1   &   1 & 0 & 0 & 1   &  $q_0$ & 0 & 1   &   0 & 1   \\
    \hline
    -  & 1 & 0 & 1   &   -     & - & -   &   - & - & - & -   &  -     & - & -   &   - & -   \\
    -  & 1 & 1 & 0   &   -     & - & -   &   - & - & - & -   &  -     & - & -   &   - & -   \\
    \hline
    \multicolumn{16}{|c|}{\centering\Large{\ldots}} \\
    \hline
    -  & 1 & 1 & 1   &   -     & - & -   &   - & - & - & -   &  -     & - & -   &   - & -   \\
    \hline
\end{tabularx}
\end{table}

\newenvironment{nscenter}
 {\parskip=0pt\par\nopagebreak\centering}
 {\par\noindent\ignorespacesafterend}

Проведем минимизацию полученных СДНФ с помощью карт Карно.

\addanonimghere{res/karn-tables.png}{1}{}{}

%\begin{table}[H]
%    \centering
%    \subfloat[\Large{$y_2$}]{
%    \begin{tabular}{|c|cccc|}
%        \hline
%        & 00 & 01 & 11 & 10 \\
%        \hline
%        000 & 1 & 1 & 1 & 1 \\
%        001 & 1 & 1 & 1 & 1 \\
%        011 & 1 & 1 & 1 & 1 \\
%        010 & 1 & 1 & 1 & 1 \\
%        110 & - & - & - & - \\
%        111 & - & - & - & - \\
%        101 & - & - & - & - \\
%        100 & 0 & 0 & 0 & 0 \\
%        \hline
%    \end{tabular}
%    }
%    \subfloat[\Large{$y_1$}]{
%    \begin{tabular}{|c|cccc|}
%        \hline
%        & 00 & 01 & 11 & 10 \\
%        \hline
%        000 & 0 & 0 & 0 & 0 \\
%        001 & 0 & 0 & 0 & 0 \\
%        011 & 0 & 0 & 0 & 0 \\
%        010 & 0 & 0 & 0 & 0 \\
%        110 & - & - & - & - \\
%        111 & - & - & - & - \\
%        101 & - & - & - & - \\
%        100 & 1 & 0 & 0 & 1 \\
%        \hline
%    \end{tabular}
%    }
%    \subfloat[\Large{$y_0$}]{
%    \begin{tabular}{|c|cccc|}
%        \hline
%        & 00 & 01 & 11 & 10 \\
%        \hline
%        000 & 0 & 0 & 0 & 0 \\
%        001 & 0 & 0 & 0 & 0 \\
%        011 & 0 & 0 & 0 & 0 \\
%        010 & 0 & 0 & 0 & 0 \\
%        110 & - & - & - & - \\
%        111 & - & - & - & - \\
%        101 & - & - & - & - \\
%        100 & 1 & 0 & 1 & 0 \\
%        \hline
%    \end{tabular}
%    }
%\end{table}
%\begin{table}[H]
%    \centering
%    \subfloat[\Large{$D_1$}]{
%    \begin{tabular}{|c|cccc|}
%        \hline
%        & 00 & 01 & 11 & 10 \\
%        \hline
%        000 & 0 & 0 & 1 & 1 \\
%        001 & 0 & 1 & 1 & 1 \\
%        011 & 0 & 0 & 0 & 0 \\
%        010 & 0 & 1 & 1 & 1 \\
%        110 & - & - & - & - \\
%        111 & - & - & - & - \\
%        101 & - & - & - & - \\
%        100 & 0 & 0 & 0 & 0 \\
%        \hline
%    \end{tabular}
%    }
%    \subfloat[\Large{$D_0$}]{
%    \begin{tabular}{|c|cccc|}
%        \hline
%        & 00 & 01 & 11 & 10 \\
%        \hline
%        000 & 0 & 1 & 1 & 0 \\
%        001 & 0 & 1 & 1 & 0 \\
%        011 & 0 & 0 & 0 & 0 \\
%        010 & 0 & 0 & 0 & 0 \\
%        110 & - & - & - & - \\
%        111 & - & - & - & - \\
%        101 & - & - & - & - \\
%        100 & 1 & 1 & 1 & 1 \\
%        \hline
%    \end{tabular}
%    }
%\end{table}
Итоговая система логических уравнений:
\[
    \begin{array}{l c l c l c l c l}
        D_0 &=& Q_0\Bar{x}_2\Bar{x}_1 &\lor& x_2 &&&& \\[3ex]
        D_1 &=& Q_1\Bar{x}_2\Bar{x}_1 &\lor& Q_0\Bar{x}_2\Bar{x_1}x_0 &\lor&
                Q_0x_1\Bar{x}_0 &\lor& Q_1x_1\Bar{x}_0 \\[3ex]
            %&=& \Bar{x}_2\Bar{x}_1(Q_1 \lor Q_0x_0) \lor x_1\Bar{x}_0(Q_0 \lor Q_1) &&&&&& \\[5ex]
        y_0 &=& \Bar{Q}_1\Bar{Q}_0x_2 &\lor& Q_1Q_0x_2 &&&&\\[3ex]
        y_1 &=& \Bar{Q}_0x_2 &&&&&& \\[3ex]
        y_2 &=& \Bar{x}_2    &&&&&&
    \end{array}
\]

На основе этой системы можно построить логическую схему устройства:
\addimghere{res/circuit.png}{1}{}{}

Используя схему выполним тестирование автомата на последовательностях 231\$ и 111\$:

\begin{table}[H]
    \centering
    \begin{tabular}{|c|cc|c|cc|}
        \hline
        Вход & $Q_1$ & $Q_0$ & Выход & $D_1$ & $D_0$ \\
        \hline
        2  & 0 & 1 & i & 1 & 0 \\
        3  & 1 & 0 & i & 0 & 0 \\
        1  & 0 & 0 & i & 0 & 0 \\
        \$ & 0 & 0 & 3 & 0 & 1 \\
        \hline
    \end{tabular}
    \hfill
    \begin{tabular}{|c|cc|c|cc|}
        \hline
        Вход & $Q_1$ & $Q_0$ & Выход & $D_1$ & $D_0$ \\
        \hline
        1  & 0 & 1 & i & 1 & 1 \\
        1  & 1 & 1 & i & 1 & 1 \\
        1  & 1 & 1 & i & 1 & 1 \\
        \$ & 1 & 1 & 1 & 0 & 1 \\
        \hline
    \end{tabular}
    \hfill
\end{table}

