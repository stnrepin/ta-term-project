% vim: set textwidth=120:

\section{Абстрактный синтез}

Будем использовать следующие алфавиты:
\[
    \begin{array}{c lc l }
        &A_\tc{вх}  &=& \left\{ 0, 1, 2, 3, \$ \right\} \\
        &B_\tc{вых} &=& \left\{ 0, 1, 2, 3, \tc{i} \right\}
    \end{array}
\]

\textcolor{red}{Составим информативное дерево}

Построим таблицу входов и выходов для автомата Мили:
\begin{table}[h]
    \centering
    \begin{tabularx}{\textwidth}{|C|C|C|C|C|}
        \hline
            & $q_0$ &  $q_1$ &  $q_2$ &  $q_3$ \\
        \hline
        0   & \slashbox{$q_0$ }{i} & \slashbox{$q_1$}{i} & \slashbox{$q_2$}{i} & \slashbox{$q_3$ }{i} \\
        \hline
        1   & \slashbox{$q_1$ }{i} & \slashbox{$q_1$}{i} & \slashbox{$q_2$}{i} & \slashbox{$q_3$ }{i} \\
        \hline
        2   & \slashbox{$q_2$ }{i} & \slashbox{$q_2$}{i} & \slashbox{$q_2$}{i} & \slashbox{$q_3$ }{i} \\
        \hline
        3   & \slashbox{$q_3$ }{i} & \slashbox{$q_3$}{i} & \slashbox{$q_3$}{i} & \slashbox{$q_3$ }{i} \\
        \hline
        \$  & \slashbox{$q_0$ }{0} & \slashbox{$q_0$}{1} & \slashbox{$q_0$}{2} & \slashbox{$q_0$ }{3} \\
        \hline
    \end{tabularx}
\end{table}

В соответствии с таблицей составим диаграмму автомата:

\begin{tikzpicture}[->,>=stealth',shorten >=1pt,auto,node distance=7cm,
                    semithick]
    \tikzstyle{every state}=[minimum size=1.7cm]

    \node[state] (q2)                         {$q_2$};
    \node[state] (q1) [left of=q2]            {$q_1$};
    \node[state] (q3) [right of=q2]           {$q_3$};
    \node[state] (q0) [above of=q2]           {$q_0$};

    \path   (q0) edge [bend left=10]          node {$1  / \tc{i}$}      (q1)
                 edge [bend left=25]          node {$2  / \tc{i}$}      (q2)
                 edge [bend left=10]          node {$3  / \tc{i}$}      (q3)
                 edge [loop left, above]      node {$0  / \tc{i}$}      (q0)
                 edge [loop right, above]     node {$\$ / 0$}           (q0)

            (q1) edge [bend left=10]          node {$\$ / 1$}           (q0)
                 edge [bend left=10]          node {$2  / \tc{i}$}      (q2)
                 edge [bend right=40,below]   node {$3  / \tc{i}$}      (q3)
                 edge [loop below]            node {$0,1 / \tc{i}$}     (q1)

            (q2) edge [bend left=25]          node {$\$ / 2$}           (q0)
                 edge                         node {$3 / \tc{i}$}       (q3)
                 edge [loop below]            node {$0,1,2 / \tc{i}$}   (q2)

            (q3) edge [bend left=10]          node {$\$ / 3$}           (q0)
                 edge [loop below]            node {$0,1,2,3 / \tc{i}$} (q3);

\end{tikzpicture}
%
Минимизируем автомат Мили. Для этого найдем все эквивалентные состояния:
\begin{table}[h]
    \centering
    \begin{tabularx}{\textwidth}{|C| C C C}
        \cline{1-2}
        $q_1$ & \multicolumn{1}{C|}{\large{$\times$}} &                                                                               \\
        \cline{1-3}
        $q_2$ & \multicolumn{1}{C|}{\large{$\times$}} & \multicolumn{1}{C|}{\large{$\times$}} &                                       \\
        \cline{1-4}
        $q_3$ & \multicolumn{1}{C|}{\large{$\times$}} & \multicolumn{1}{C|}{\large{$\times$}} & \multicolumn{1}{C|}{\large{$\times$}} \\
        \hline
              & \multicolumn{1}{C|}{$q_0$}            &  \multicolumn{1}{C|}{$q_1$}           & \multicolumn{1}{C|}{$q_2$}            \\
        \hline
    \end{tabularx}
\end{table}

Из таблица видно, что в автомате отсутствуют эквивалентные состояния, то есть он уже минимален.

